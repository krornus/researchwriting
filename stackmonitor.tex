\setlength{\parindent}{10ex}

\documentclass{article}

\title{Vulnerability Auditing Through Live Stack Tracing} \author{Spencer
Powell} \date{\today}

\begin{document}

\maketitle

\section {Overview}

%Make a workflow figure clarify relationship between debugger and monitor which
%preceeds which?

The workflow of the program is as follows. The program takes a binary file,
which is sent to the binary debugger. The debugger will call functions within
the program itself, in an effort to lessen the amount of code executed and
increase the path coverage. The debugger is capable of giving arguments to the
given function, which can be randomized. Previously known fuzzing techniques can
be applied to these functions, in an effort to trigger crashes. Fuzzing is a
method of crafting random inputs for a program in an effort to explore all paths
within a given binary or function. Exploring all paths through a givent binary
is an important step in this process because a given vulnerability in a binary
may not be located within a commonly executed logic path. The stack monitor
maintains a log of stack activity during each execution, which is written to a
file. Each time the fuzzer provides a different input to the program, a new log
file is produced and saved. This collection of files is then sent to the
vulnerability detector. The detector uses machine learning to recognize
vulnerable patterns in stack activity, and tags potentially vulnerable sections
of machine code.

\section{The Stack Monitor}

    Our stack monitor serves to monitor the execution of an arbitrary program and
then creates a log of all the stack activities, which is later analyzed for
vulnerability detection.  It takes as input a binary executable of a program
(currently only 64 bit x86 ELF's are supported), and any command line arguments.
The monitor produces a sequence of stack read and write operations which can be
accessed through python. It also produces an instruction trace of the program,
and information about the location and size of the stack in memory per
instruction.

    The monitor operates by making use of binary instrumentation. Binary
instrumentation works by injecting compiled code into the program. The compiled
instrumentations are dynamically injected after each binary instruction of the
program. The instrumentations aim to collect information from the program
registers and capture relevant data to stack reads and writes. Specifically, we
capture the following registers which are later stored in a log file: the stack
pointer (SP), the base pointer for the stack (BP), and the instruction pointer
(IP). Other data is also captured from the program's environment. If the
currently instrumented instruction is a read or write to memory, the code checks
if the address is within the range of the program's stack. If the effective
address is within the correct range for the stack, then the data being read or
written is saved, along with the memory address and the length of the data which
is read or written. All saved information is then sent to the stack monitor
using a client-server model for interprocess-communication. Currently the
information generated by the stack monitor is then placed into a log file,
illustrated in Figure 1.2.

% TODO: create figure for stack monitor

    The logic of checking addresses, reading memory, reading registers, and
sending data to a server is executed per instruction for any given child binary,
which can significantly reduce the performance of code. This reduction of
efficiency is partially solved through the binary debugger.

\section{The Binary Debugger}

% better organize into application,input,output
The binary debugger was created in order to reduce the number of instructions
which are executed while running a binary.
% TODO: Make sure this is the actual process we use?
The debugger is spawned by the stack monitor which in turn calls specified
functions. Because we are only calling individual functions instead of the
entire program, the execution cost is greatly reduced. This is important,
because the next stage of the process, path exploration, requires multiple runs
of a given program. When the program completes, the stack monitor outputs the
log as usual, but only for the sections which are executed by the debugger.

% revise describe use of debugger in application, input, produce
% non-relevant technical details?
The binary debugger works by making use of the ptrace system call. Ptrace grants
a parent application the ability to monitor and edit the memory of an active
child process. The debugger can set breakpoints in a child, and call arbitrary
functions that exist within the child. The function calls can be constructed
with arguments, but no initialization is done to global data.  This can be a
potential issue for processes which rely on global variables or preallocated
memory. Fake allocations can be performed by manually allocating space on the
stack before calling, but currently the only fix for global variables is to
partially execute sections of the program before executing the desired code.


\section {Path Exploring}
Dynamic execution has a few issues with it that must be dealt with. One of the
biggest issues with dynamic execution traces is that of code coverage.
Identifying a vulnerability requires that you first execute the vulnerable piece
of code. We have elected to use fuzzing as a technique for code coverage, as it
is well researched and there are many pre-existing methods with which to explore
a binary.

\section {The Vunerability Detection}
The data from the monitor is analyzed through machine learning, the data must be
preprocessed.  The addresses of the stack operations are made to be relative to
the base pointer, and the stack frame is given in an overall size instead of
both a stack and base pointer.

\section{Key Technical Challenge}

The key technical challenge of this tool is handling the massive amount of data.
Per instruction, the log contains three addresses, and extra optional data for
memory operations on the stack. This pressents two major issues. The first issue
is the massively increased runtime of the target binary which is required in
order to generate the information, and the second issue is comprehending and
manipulating the massive amount of data which is created.

The large influx of data is handled in two distinct methods. The first method is
a binary debugger which attempts to extract and call a given function from a
binary. This allows us to limit our scope of analysis to a single area, and
create multiple logs based on differing inputs. The second method used to handle
the large dataset is machine learning.  Our goal is to discover patterns in
stack behavior could indicate vulnerable logic in a binary.  There is far too
much data for this to be done manually, so machine learning was applied.

\end{document}

\setlength{\parindent}{10ex}

\documentclass{article}

\title{Vulnerability Auditing Through Live Stack Tracing} \author{Spencer
Powell} \date{\today}

\begin{document}

\maketitle

\section{Stack Monitor}
\par
\ \ \ \ The stack monitor is a tool that was
written in order to create logs of stack activity for arbitrary programs,
implying the vulnerability auditing will be dynamic, instead of static. Dynamic
execution has a few issues with it that must be dealt with. One of the biggest
issues with dynamic execution traces is that of code coverage. Identifying a
vulnerability requires that you first execute the vulnerable piece of code. We
have elected to use fuzzing as a technique for code coverage, as it is well
researched and there are many pre-existing methods with which to explore a
binary. \par
    The monitor takes a binary executable of a program (currently only
64 bit x86 ELF's are supported), and any command line arguments. The monitor
produces a sequence of stack read and write operations which can be accessed
through python. It also produces an instruction trace of the program, and
information about the location and size of the stack in memory per instruction.
The monitor works by making use of binary instrumentation.  Binary
instrumentation works by injecting compiled code into the program. The code for
our project is injected on a per instruction basis into the program.  This can
greatly increase the runtime of code, due to instruction count being multiplied
by a large factor. Currently the information created by the stack monitor is
placed into a log, illustrated in Figure 1.1.  Because the data from the monitor
is analyzed through machine learning, the data must be preprocessed.  The
addresses of the stack operations are made to be relative to the base pointer,
and the stack frame is given in an overall size instead of both a stack and base
pointer.

\section{Key Technical Challenge}

The key technical challenge of this tool is handling the massive amount of data.
Per instruction, the log contains three addresses, and extra optional data for
memory operations on the stack. This pressents two major issues. The first issue
is the massively increased runtime of the target binary which is required in
order to generate the information, and the second issue is comprehending and
manipulating the massive amount of data which is created. \par
    The large influx of data is handled in two distinct methods. The first method
is a binary debugger which attempts to extract and call a given function from a
binary. This allows us to limit our scope of analysis to a single area, and
create multiple logs based on differing inputs. The second method used to handle
the large dataset is machine learning.  Our goal is to discover patterns in
stack behavior could indicate vulnerable logic in a binary.  There is far too
much data for this to be done manually, so machine learning was applied.

\section{Workflow}

The workflow of the program is as follows. The program takes a binary file,
which is sent to the stack monitor. The stack monitor spawns a binary debugger,
which will call functions within the program itself. Each function run is
monitored and the output is stored in a log. The debugger is capable of giving
arguments to the given function, which can be randomized. Previously known
fuzzing techniques can be applied to these functions, in an effort to trigger
crashes.

\end{document}

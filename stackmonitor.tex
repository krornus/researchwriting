\setlength{\parindent}{10ex}

\documentclass{article}
\usepackage{graphicx}

\title{Vulnerability Auditing Through Live Stack Tracing} \author{Spencer
Powell} \date{\today}

\begin{document}

\maketitle

\section {Overview}

%Make a workflow figure clarify relationship between debugger and monitor which
%preceeds which?

%TODO: first make the graph; then we will explain it.

The overall workflow of our vulnerability detection toolkit is shown in
%Figure\ref{fig-workflow}
Figure 1.1 and includes four main components: the stack monitor, which monitors
and logs stack activiy, the debugger, which calls functions within a target
program, a fuzzer, which supplies constructed random inputs to the program, and
a pattern detector, which uses machine learning to analyze the newly created
stack monitor logs and detect vulnerable code patterns. After the stack monitor
invokes the debugger, the debugger will call functions within the program
itself, in an effort to lessen the amount of code executed and increase the
path coverage. The debugger is capable of giving arguments to the given
function, which can be randomized by fuzzing. Fuzzing is a method of crafting
random inputs for a program in an effort to explore all paths within a given
binary or function. Previously known fuzzing techniques can be applied to these
functions, in an effort to trigger crashes. Exploring all paths through a
givent binary is an important step in this process because a given
vulnerability in a binary may not be located within a commonly executed logic
path. The stack monitor maintains a log of stack activity during each
execution, which is written to a file. Each time the fuzzer provides a
different input to the program, a new log file is produced and saved. This
collection of files is then sent to the vulnerability detector. The detector
uses machine learning to recognize vulnerable patterns in stack activity, and
tags potentially vulnerable sections of machine code.

\begin{figure}
\begin{center}
%\vspace{-0.3in}
\includegraphics[height=4in]{workflow.png}
\end{center}
%\vspace*{-0.3in}
\caption{The Overall Workflow}
%\vspace{-0.4in}
\label{fig-workflow}
\end{figure}

\section{The Stack Monitor}

Our stack monitor serves to monitor the execution of an arbitrary program and
then creates a log of all the stack activities, which is later analyzed for
vulnerability detection. It takes as input a binary executable of a program
(currently only 64 bit x86 ELF's are supported). The monitor dynamically
instruments the user executable to log down the values of the runtime stack
pointer (SP), the base pointer for the stack (BP), and the instruction pointer
(IP) before each instruction is executed. If the instruction reads or writes to
memory, instrumentation is inserted to check whether the targeted memory is
within the runtime stack, and if yes, the memory operation is logged. All the
logged information is sent to the stack monitor from the instrumented binary
executable using a client-server model for interprocess-communication.
%An example of the logged information for ??? is illustrated in
%Figure~\ref{fig-logfile}.
The stack monitor can be configured to either output all the logs into an
external file, or in an internal data structure that can be accessed through
python.

% TODO: create figure for stack monitor

\section{The Binary Debugger}

The binary debugger provides an programming interface for the developer to invoke the stack monitor 
to track any selected regions of code.  TODO: what is the interface (API)? make a figure and reference the figure. Then try to explain the logisitics of the API.


TODO: provide an example of using the API of debugger in the fuzzer. Also the figure will be used to explain the fuzzer. 


The programming interface in Figure?? is implemented by making system call to TODO: the specific ptrace call (with interface), TODO: explain how the ptrace interface work. In particular, the debugger invokes ptrace with the TODO {what requests} to allow the user to monitor and edit the memory of any child process, 
to set breakpoints in a child, and to call any function in the child. 

The function calls can be
constructed with arguments, but no initialization is done to global data. This
can be a potential issue for processes which rely on global variables or
preallocated memory. Fake allocations can be performed by manually allocating
space on the stack before calling, but currently the only fix for global
variables is to partially execute sections of the program before executing the
desired code.


\section {Path Exploring} Dynamic execution has a few issues with it that must
be dealt with. One of the biggest issues with dynamic execution traces is that
of code coverage. Identifying a vulnerability requires that you first execute
the vulnerable piece of code. We have elected to use fuzzing as a technique for
code coverage, as it is well researched and there are many pre-existing methods
with which to explore a binary.

\section {The Vunerability Detection} The data from the monitor is analyzed
through machine learning, the data must be preprocessed. The addresses of the
stack operations are made to be relative to the base pointer, and the stack
frame is given in an overall size instead of both a stack and base pointer.

\section{Key Technical Challenge}

The key technical challenge of this tool is handling the massive amount of
data. Per instruction, the log contains three addresses, and extra optional
data for memory operations on the stack. This pressents two major issues. The
first issue is the massively increased runtime of the target binary which is
required in order to generate the information, and the second issue is
comprehending and manipulating the massive amount of data which is created.

The large influx of data is handled in two distinct methods. The first method
is a binary debugger which attempts to extract and call a given function from a
binary. This allows us to limit our scope of analysis to a single area, and
create multiple logs based on differing inputs. The second method used to
handle the large dataset is machine learning. Our goal is to discover patterns
in stack behavior could indicate vulnerable logic in a binary. There is far too
much data for this to be done manually, so machine learning was applied.

\end{document}
